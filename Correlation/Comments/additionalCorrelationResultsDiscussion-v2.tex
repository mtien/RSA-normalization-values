\documentclass[12pt]{article}

\usepackage[margin=2.5cm]{geometry}
\usepackage{graphicx}
\usepackage{lscape}
\usepackage{cite}
\usepackage{amsmath}
\usepackage{color}
\usepackage{verbatim}
\renewcommand{\bottomfraction}{.9}
\renewcommand{\topfraction}{.9}
\renewcommand{\textfraction}{0.1}
\renewcommand{\floatpagefraction}{.9}

\linespread{1.5}
\begin{document}

\section{Results}
We further evaluated empirically derived hydrophobicity scales with other experimental scales: \cite{Fauchere1983}, \cite{Radzicka1988}, \cite{Wimley1996}, \cite{MacCallum2007}, and \cite{Moon2011}. Unlike the correlations from \cite{Wolfenden1981} and \cite{Kyte1981}, we found that these five experimental scales showed a different pattern in their correlations. 
\cite{Fauchere1983} showed the best correlation with our mean RSA scales (r=.91) than \cite{Rose1985} (r=.90) and did much worse with its correlation with our 100% buried scale (r=.73). \cite{Radzicka1988} correlations with our scales showed the same improvements that we saw in the \cite{Wolfenden1981}. The 95% buried correlation with the \cite{Radzicka1983} (r=.90) was much better than that of \cite{Rose} scale (r=.80); however, the 100% buried scale did not correlate as well (r=.85). \cite{Wimley1996} showed the worst correlation thus far with all of our scales and \cite{Rose1985} (r=.43). Unlike the analysis with our other scales, the 100% buried scale's correlation with the Wimley scale did the worse (r=.32) and the  mean and median scales did the best (r=.48). \cite{MacCallum} scale's correlation with our scales were better than the \cite{Rose1985} scale (r=.80); the best ones were the transformed Box-Cox mean and the 95% buried scale (r=.90). Lastly the \cite{Moon2011} scale's correlations with our scales was still better than \cite{Rose1985} scale's correlation (r=.70); however, the correlation with the 95% buried scale (r=.78) did much better than the 100% buried scale (r=.69).
Overrall, improved correlations between experimental and empirical scales were observed in the 95% buried scales. 
\section{Discussion}
Correlations between experimental hydrophobicity scales with our empirically derived scales demonstrate improved empirical hydrophobicity scales using our normalization values. In every single comparison, our mean RSA scales do better than \cite{Rose1985} mean RSA hydrophobicity scales. All experimental scales seemed to correlate pretty well with our empirically derived scales with the exception of the \cite{Wimley} hydrophobicity scale. \cite{Wimley} performed a partioning experiment with pentapeptide species, Ace-WLxLL with x being one of the naturally occuring 20 amino acids, between water and 1-octanol, simlar to the set up of \cite{Fauchere1983}. Using a pentapeptide, rather than single amino acids, the \cite{Wimley} hydrophobicity scale does not reflect the hydrophobic character of just an amino acid itself, but the $K_d$ of a pentapeptide. It is possible to adjust the effects of occulsion by neighboring residues, which would provide a better picture of a single amino acids contribution to a peptide segments hydrophobic character. 
The 
This analysis shows that there is a strong correlation between the tendency of an amino acid to be buried or exposed and its hydrophobic character. Resolution between experimental and empirical data provides further evidence of our current understanding of protein energetics. In order to better model protein thermodynamics, experimental hydrophobic measurements need to be verified by what we observe in our current database of known protein structures. This analysis provides good evidence that the distrubtion of amino acids within a globular protein is consistent with the basic chemistry of amino acids. 


\cite{Fauchere1983} hydrophobicity scale was derived from the free energies of transfer of acetyl amino amides between octanol and water. \cite{Radzicka1988} hydrophobicity scales were derivied from the free energies of transfer of amino acid anologs between cyclohexane and water. \cite{MacCallum2007} hydrophobicity scale was derived from molecular dynamic simulations that calculated the distribution of \cite{Radzicka1983} side chain analogs wthin a bilayer. \cite{Moon2011} hydrophobicity scale was based the in vitro equilibrium between the folded and unfolded state of a mutated membrane-inserted protein, outer membrane phospholipase A. 

\bibliographystyle{pnas}
\bibliography{bibliographyNew}


\end{document}